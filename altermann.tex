\documentclass[a4paper,12pt,notitlepage]{article}
\usepackage[T2A]{fontenc}
\usepackage[utf8]{inputenc}
\usepackage[russian]{babel}
%
\usepackage{fullpage}

\usepackage{color}
\usepackage[table]{xcolor}
\usepackage{etoolbox}
\usepackage{mathtools}
\usepackage{multicol}
\usepackage{verse}
\usepackage{hyperref}
\usepackage{minted}

\setlength{\parskip}{0.7em}

\hypersetup{
    pdftitle={Alter Mann},
    colorlinks=true,
    linkcolor=blue,
    citecolor=blue,
    filecolor=magenta,      
    urlcolor=cyan,
}

\author{Дмитрий Григорьев}
\title{Граф взаимоотношений\\персонажей песни <<Alter Mann>>\\группы Knorkator}

\renewcommand{\listingscaption}{Листинг}
\renewcommand{\listoflistingscaption}{Список листингов}

\begin{document}
  \maketitle
  В песне <<Alter Mann>> группы Knorkator представлены сложные взаимоотношения различных персонажей и индифферентное отношение лирического героя песни (<<Старика>>) к ним. Текст песни на немецком языке \cite{knorkator01}:
  
  \poemtitle{Alter Mann}
  \settowidth{\versewidth}{Doch Pech denn er ist schon länger}
  \begin{verse}[\versewidth]
	  \flagverse{1.}Daniela steht auf Jonas, \\
	  Doch Jonas liebt Vanessa, \\
		Vanessa wär gern mit Lars zusammen, \\
		Doch der findet Melanie besser. \\
		Die allerdings steht eher auf Tim, \\
		Tim wiederum findet Jennifer cool, \\
		Jennifer jedoch ist verliebt in Kevin, \\
		Aber Kevin ist schwul. \\!
		Tobias ist scharf auf Annika, \\
		Und Annika eigentlich auch auf ihn, \\
		Doch Pech denn er ist schon länger \\
		der Schwarm ihrer besten Freundin. \\!


		\flagverse{\textsc{chorus}}Zum Glück bin ich ein alter Mann, \\
		Das geht mich alles nichts mehr an, \\
		Bin weder Hetero noch schwul, \\
		Und mir genügt ein angenehmer Stuhl. \\!

		\flagverse{2.}Hagen erschießt Sebastian, \\
		Sven überfährt Hagen, \\
		Christopher erwürgt Sven \\
		Und wird von Yannick erschlagen. \\
		Björn und Jochen ermorden Yannick, \\
		Lukas erledigt Björn und Jochen, \\
		Benjamin macht Lukas alle, \\
		Und wird von Nils erstochen. \\!
		Nils wird daraufhin abgemurkst, \\
		Von Felix und Thorsten, \\
		Alexander killt Thorsten und Felix, \\
		Und wird erschossen. \\!

		\flagverse{\textsc{chorus}}Zum Glück bin ich ein alter Mann, \\
		Das geht mich alles nichts mehr an, \\
		Auf dem Balkon hab ich es gut, \\
		Und schau was ihr da unten alles tut. \\!

		\flagverse{\textsc{break}}Ich hab eh nicht mehr lange, \\
		In diesem irdischen Jammertal, \\
		Schlagt doch alles in Stücke, \\
		Es ist mir so egal. \\!
		
		\flagverse{\textsc{chorus}}Zum Glück bin ich ein alter Mann, \\
		Das geht mich alles nichts mehr an, \\
		Und wenn die Erde explodiert, \\
		Entschuldigt dass es mich nicht interessiert. \\
		Entschuldigt dass es mich nicht interessiert! \\
  \end{verse}
  
  В первом куплете описаны любовные отношения ряда персонажей, во втором куплете --- вражда. Весь остальной текст выражает отношение к происходящему лирического героя. После внимательного прослушивания песни и анализа текста, мы можем построить следующий социальный граф (\ref{lst:dot}):
  
    \begin{listing}[H]
      \caption{Graphviz-файл \texttt{altermann.dot}}
      \label{lst:dot}
      \inputminted[linenos]{dot}{altermann.dot}
    \end{listing}
    
    В результате обработки исходного кода графа программой \texttt{dot}, входящей в пакет \href{https://http://www.graphviz.org/}{Graphviz}, после выполнения команды \mint{bash}|dot -Tpng altermann.dot > altermann.png| получаем изображение \ref{fig:graph}.
    
    
    \begin{figure}[ht]
      \centering
      \includegraphics[width=8cm]{altermann.png}
      \caption{Социальный граф, построенный по тексту песни <<Alter Mann>>}
      \label{fig:graph}
    \end{figure}
    
    \newpage
    \listoflistings
    \listoffigures
    \begin{thebibliography}{1}
      \bibitem{knorkator01}
      Knorkator --- <<Alter Mann>>, (\href{https://genius.com/Knorkator-alter-mann-lyrics}{genius.com}), Das nächste Album aller Zeiten, NB 1779-0, 2007

\end{thebibliography}
    
    
  
      
\end{document}